\section{Powiązane prace}
\label{sec:related-works}

Dla alfabetów migowych, najczęściej dostępne zbiory danych składają się z ogromnej liczby zdjęć, co było już z góry ogromnym ograniczeniem. Autorzy przeważnie pomijają kompletnie znaki dynamiczne albo zakłamują dane, dopuszczając zdjęcie dłoni w stałej pozycji. Modele trenowane na takich danych za dużo nie wnoszą, z racji, że nie uwzględnia się ruchu, a schematycznie ułożone piksele na zdjęciach, które są klasyfikowane jako dana litera. Nie da się na tym zbudować całego słownictwa, poza literowaniem tylko części słów. Tworzenie modelu do klasyfikacji statycznych liter PJM nie byłoby w pełni przydatne, ponieważ alfabet PJM składa się głównie z liter dynamicznych. Klasyfikacja statycznych liter miałaby niewiele praktycznego zastosowania, kiedy danymi wejściowymi są po prostu różne układy pikseli. Celem projektu było stworzenie kompleksowego modelu, który mógł obsługiwać zarówno statyczne, jak i dynamiczne litery znaków, co jest bardziej przydatne w kontekście języka migowego.

Na potrzeby udowodnienia konceptu stworzono autorski zbiór danych liczący 36000 filmów dwusekundowych. Najbardziej pracochłonną częścią projektu była organizacja czasu dla osób, aby skompletować zbiór w całość oraz weryfikacja nagrań, aby ewentualnie móc nagrać na nowo materiały w razie pomyłek. Jak na ilość wykonanej pracy, zbudowano bardzo mały zbiór danych na tle ilości znaków, jakie występują w zasobie języka. Choć istnieje wiele aktywnie rozwijanych projektów wykorzystujących uczenie maszynowe w lingwistyce, temat języka migowego nie jest jeszcze dobrze zbadany na tyle, aby móc budować takie rozwiązania jak rozpoznawanie mowy, a w przypadku języka migowego interpretacja wypowiedzi składających się ze znaków. Można zauważyć, że w badaniach powielają się podobne problemy, które są najczęściej związane z zestawem danych.

Obecnie jest bardzo dużo publikacji, których celem jest rozwój komunikacji między światem Głuchych i słyszących. W 2023 badacze Microsoftu przeprowadzili szczegółową analizę wpływu szumu informacyjnego wśród osób słyszących w detekcji języka migowego na dane treningowe oraz testowe~\cite{pal2023}. Najnowsze modele detekcji zapewniają wysoką wydajność, ale wciąż jest zapotrzebowanie na dobrze etykietowane dane, które uwzględniają obecność nadmiaru informacji, aby zapewnić właściwą ocenę. Kolejną kwestią jest samo zrozumienie funkcjonowania języka. Obecne metody nie uwzględniają w pełni możliwości języka takich jak mimika, pauzy czy też długość trwania znaku. Poprawne zrozumienie semantyki języka jest kluczowe do tworzenia modeli detekcji i metod segmentacji~\cite{deSisto2021}. O ile jest to łatwiejsze w przypadku interpretacji mowy ze ścieżki dźwiękowej, to niestety nie ma to przełożenia na obraz wideo, z racji, że nie da się przedstawić języka migowego na spektrogramie.
