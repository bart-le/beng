\hypersetup{pageanchor=true}

\chapter{Podsumowanie}\label{ch:summary}

W chwili publikacji niniejszej pracy, jednym z głównych ograniczeń była ilość dostępnych danych oraz ich uniwersalność. Klasyfikacja szerokiego zakresu danych stanowi wyzwanie, ponieważ istotne jest filtrowanie danych wejściowych, aby usunąć wyłącznie nadmiar informacji. Wyzwaniem jest postawienie granicy między szumem informacji a rzeczywistą treścią w komunikacji. Konieczne jest uwzględnienie wszelkich czynników, które mogą wpłynąć na zrozumienie przekazu przez osobę Głuchą. Eliminowanie zbędnych danych jest ważne, ale równie istotne jest zrozumienie struktury samego języka migowego.

Język migowy opiera się głównie na sekwencjach położenia rąk, co generuje znacznie więcej kombinacji niż statyczne znaki, których jest mniej. W badaniach interpretacji języka migowego często stosuje się bezpośrednie tłumaczenie pojedynczych znaków. Ważne jest dążenie do unifikacji danych, aby były łatwe do interpretacji. W dziedzinie lingwistyki migowej opracowano wiele systemów notacji wizualnych, które często powstawały w celach badawczych. Obecnie na świecie istnieje wiele korpusów języków migowych, które są rozwijane, a praca nad etykietowaniem danych jest kontynuowana. Powstaje też wiele algorytmów, które pomagają przekształcić notacje, takie jak HamNoSys, na klasy odpowiednie do uczenia maszynowego~\cite{majchrowska2022}.

W bardziej zaawansowanym etapie rozwoju tej dziedziny można rozważyć transkrypcję ruchu na system notacji, co pozwoliłoby na interpretację języka migowego w kontekście przetwarzania języka naturalnego. Jednak istnieją trudności związane z określeniem progu błędu w interpretacji położenia rąk. Systemy notacji, takie jak HamNoSys, oferują dokładny opis, ale nie zawsze prowadzi to do jednolitych danych w korpusie, czego konsekwencją są rozbieżności w danych~\cite{ferlin2023}.

Projekt, który został przedstawiony w niniejszej pracy jest jedynie namiastką możliwości modeli sztucznych sieci neuronowych w lingwistyce migowej. Zbiór danych stworzony na potrzeby tego badania uwzględniał tylko rękę prawą, zakładając, że jest to ręka dominująca osoby przed kamerą. Celem badań było rozszerzenie wiedzy na temat przetwarzania języka migowego. Projekt miał także na celu zwiększenie świadomości ukrytych wyzwań przed przed podjęciem bardziej zaawansowanych projektów i akceptacji dla osób korzystających z języka migowego w przestrzeni cyfrowej.
