\section{Polski Język Migowy w świetle prawa}
\label{sec:polish-sign-language}

W Polsce, choć temat jest znany przez sektor publiczny od co najmniej dekady, te bariery wciąż są zwykle niezauważalne dla osób słyszących, ale dla Głuchych są one bardzo trudne do pokonania w codziennej egzystencji. W 2011 roku weszła w życie znowelizowana ustawa o radiofonii i telewizji, która w artykule 18a nakłada na nadawców programów telewizyjnych obowiązek zapewniania dostępności programów przez wprowadzanie audiodeskrypcji, napisów dla niesłyszących oraz tłumaczeń na język migowy tak, aby co najmniej 10\% czasu nadawania posiadało takie udogodnienia~\cite{ustawa1992}.

Choć uchwalenie tej ustawy było ważnym krokiem w kierunku zapewnienia dostępności telewizji, Polska wciąż pozostaje daleko w tyle w porównaniu z innymi krajami Unii Europejskiej. Mimo wprowadzenia ustawy regulującej korzystanie z języka migowego oraz innych sposobów komunikacji wiele instytucji i urzędów państwowych wciąż nie zapewnia odpowiedniego dostępu do tłumaczy dla osób Głuchych. To generuje trudności lub wręcz uniemożliwia skorzystanie z usług publicznych. Programy telewizyjne również rzadko uwzględniają potrzeby osób Głuchych, a jeśli już, to bardzo rzadko są one opatrzone napisami lub tłumaczeniem na migowy. W efekcie, osoby Głuche nadal mają ograniczony dostęp do informacji publicznej. Konsekwencją marginalizacji jest niepełna realizacja praw społecznych~\cite{teper2016}.

Jednym z dowodów braku zrozumienia jest rozporządzenie Krajowej Rady Radiofonii i Telewizji z 13 kwietnia 2022~\cite{rozporzadzenie2022}. Rozporządzenie nie uwzględnia różnic w komunikacji, którymi posługują się osoby Głuche i słabosłyszące. Projekt zakłada stosowanie tylko napisów lub tłumaczenia. Nie przewiduje łącznego ich stosowania. Tekst nie jest alternatywą dla samego tłumaczenia języka migowego. Słyszący często zakładają, że każdy zna język polski w piśmie i potrafi się nim biegle posługiwać, kiedy w rzeczywistości osoby Głuche po prostu nie słyszą i nie porozumiewają się za pomocą języka mówionego. Wiele osób z niepełnosprawnością słuchu postrzega każdy język foniczny jako język obcy. Jest spora szansa wystąpienia problemów w komunikacji nawet jeśli osoby Głuche mówią płynnie po polsku. Przeciętny poziom umiejętności czytania i pisania jest znacznie niższy u osób niesłyszących niż u słyszących~\cite{perfetti2000}.
